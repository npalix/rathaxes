\documentclass[francais]{rtxarticle}

\usepackage{listings}

\lstset{%
basicstyle=\ttfamily\footnotesize\color{lightgray},
backgroundcolor=\color{black}
}

\title{Premier Pas}
\author{Louis Opter}

\rtxdoctype{Documentation Utilisateur}
\rtxdocstatus{Brouillon}
\rtxdocversion{0.1}

\rtxdochistory
{
0.1 & 09/04/2011 & Louis Opter & Traduction de la version Anglaise \\
}

\begin{document}

\maketitle

\begin{abstract}
\rtx est un Langage de programmation Spécifique à un Domaine (DSL) qui permet de
décrire des pilotes de périphériques. \rtx\ compile vers des modules noyaux
écrits en C pour Linux, Windows et OpenBSD.

Ce document explique comment installer \rtx\ sous Windows et les systèmes
d'exploitations de type Unix et comment générer votre premier pilote de
périphérique.

Vous trouverez aussi dans ce document des liens vers la documentation complète
et comment compiler la dernière la version de \rtx.
\end{abstract}

\tableofcontents

\section{Installation}

Un installateur existe pour Windows, sur les autres systèmes d'exploitation
seulement une installation depuis les sources est supportée pour le moment.
Bien sûr, si vous souhaitez développer sur \rtx\ en lui même vous pouvez
choisir d'installer \rtx\ depuis les sources (càd sans l'installateur).

Bien que \rtx\ génère des pilotes uniquement pour Windows, Linux et OpenBSD il
est possible de l'installer et de l'utiliser depuis d'autre systèmes (comme
d'autre BSD et MacOS X).

\subsection{Using the Windows installer}

L'installateur fournis pour Windows inclue tout ce dont vous avez besoin pour
écrire des pilotes en utilisant \rtx\ : le compilateur et sa documentation.

Téléchargez l'\href{http://rathaxes.googlecode.com/files/rathaxes-latest.exe}{installeur},
puis exécuter le et suivez les instructions sur l'écran.

Pour compiler et utiliser les pilotes générés par \rtx\ vous aurez besoin
d'installer le kit de développement de pilotes de périphériques de Microsoft
Windows.

\subsection{From the sources on Unix}

If you are not on Windows you will have to install \rtx\ from a ``source
release''.

You will need to have CMake >= 2.6 installed and to download a
\href{http://rathaxes.googlecode.com/files/rathaxes-src-latest.tar.gz}{\rtx\ source tarball}.

Then extract the source tarball and ``cd'' into it, finally you can install
\rtx\ using:

\begin{lstlisting}
$ mkdir build
$ cd build
$ cmake -DCMAKE_INSTALL_PREFIX=/usr/local/ -DCMAKE_BUILD_TYPE=RELEASE ..
$ sudo make install
\end{lstlisting}

\emph{You will need to be root to issue ``make install'', this example uses
``sudo'' but accord this to your local setup.}

\section{Generate your first driver}

TBD.

\section{Diving in}

TBD.

% Give links to the documentation here.

\section{Install the development version of \rtx}

You can install the latest version of \rtx\ if you need to have the latest bugs
and features or if you want to contribute to the project. This involves checking
out the current version of the project using Mercurial and to build it manually.

\subsection{Pre-requisites}

To checkout and build the project you need to install the following softwares:
\begin{itemize}
\item Mercurial >= 1.5 (you can check the version with ``hg {-}{-}version'' and
      use \href{http://tortoisehg.bitbucket.org/download/index.html}{TortoiseHg}
      on Windows);
\item Subversion (you need to install \href{http://www.sliksvn.com/en/download}{Slik
      SVN} on Windows which ships the command line executables);
\item \href{http://www.cmake.org/cmake/resources/software.html}{CMake} >= 2.6
      (you can check the version with ``cmake {-}{-}version'');
\item A compiler tool-chain (for example by installing the ``build-essential'' package on a
      Debian-like GNU/Linux distribution or by installing
      \href{http://www.microsoft.com/express/Downloads/#2010-Visual-CPP}{Visual Studio Express for C++}
      on Windows).
\end{itemize}

\subsection{Checkout the sources}

Open a shell (you can do this from the Visual Studio menu on Windows), and
checkout the project using:

\begin{lstlisting}
$ hg clone https://rathaxes.googlecode.com/hg/ rathaxes
$ cd rathaxes
\end{lstlisting}

Keep the shell open, the next section explains how to build \rtx\ on Windows or
Unix.

\subsection{Build \rtx}

If you are using an Unix like operating system use the following commands:

\begin{lstlisting}
$ mkdir build
$ cd build
$ cmake ..
$ make
\end{lstlisting}

If you are on Windows use:

\begin{lstlisting}
$ mkdir build
$ cd build
$ cmake -G "NMake Makefiles" ..
$ nmake
\end{lstlisting}

\rtxmaketitleblock

\end{document}

