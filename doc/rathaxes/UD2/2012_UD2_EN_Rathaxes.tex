\documentclass[american]{rtxreport}

\usepackage{color}
\usepackage{listings}

\usepackage[utf8]{inputenc}

\title{User documentation}
\author{Thomas Luquet}

\rtxdoctype{UD2}
\rtxdocstatus{Draft}
\rtxdocversion{0.1}

\rtxdochistory
{
0.1 & 11/03/2011 & Thomas Luquet & Initial Import \\
\hline
0.2 & 11/03/2010 & Louis Opter & Some corrections and make it compile \\
}

\definecolor{lstbackground}{rgb}{0.95, 0.95, 0.95}
\definecolor{lstcomment}{rgb}{0, 0.12, 0.76}
\definecolor{lstkeyword}{rgb}{0.66, 0.13, 0.78}
\definecolor{lststring}{rgb}{0.67, 0.7, 0.13}
\definecolor{lstidentifier}{rgb}{0.1, 0.1, 0.1}

\lstset{
        tabsize=2,
        captionpos=b,
        emptylines=1,
        frame=single,
        breaklines=true,
        extendedchars=true,
        showstringspaces=false,
        showspaces=false,
        showtabs=false,
        basicstyle=\color{black}\small\ttfamily,
        numberstyle=\scriptsize\ttfamily,
        keywordstyle=\color{lstkeyword},
        commentstyle=\color{lstcomment},
        identifierstyle=\color{lstidentifier},
        stringstyle=\color{lststring},
        backgroundcolor=\color{lstbackground}
}

\definecolor{grey}{rgb}{0.90,0.90,0.90}
\definecolor{rBlue}{rgb}{0.0,0.24,0.96}
\definecolor{rRed}{rgb}{0.6,0.0,0.0}
\definecolor{rGreen}{rgb}{0.0,0.4,0.0}

\lstdefinelanguage{rathaxes}%
{
	morekeywords={DECLARE, SEQUENCE, INTERFACE, IMPLEMENTATION, FROM, READ,
        OPTIONAL, CONFIGURATION_VARIABLE, USE, AS, WITH, SEQUENCES, ON, ELSE,
        LET, PROVIDES, REQUIRE, THROWS, FINALLY, FOREACH, IN, AND, OR, THROW,
        HANDLE_ERROR, NOT, REGISTER, LIKE, BIT, INTEGER, DOUBLE, BOOLEAN,
        STRING, MAPPED_AT, PCI},%
	sensitive=true,%
	morecomment=[l][\color{rRed}]{//},%
	morecomment=[l][\color{rRed}]{\#},%
	morecomment=[s][\color{rRed}]{/*}{*/},%
	morestring=[b][\color{rGreen}]",%
	morestring=[b][\color{rGreen}]',%
	keywordstyle={\color{rBlue}}%
}[keywords,comments,strings]
\begin{document}

\maketitle

\begin{abstract}

This document is the user documentation for the \emph{\rtx\ 2012} project.

It includes a guide and an example to install and use \rtx. It contains also a
description of the language.

\rtx\ is a multi-platform peripheral driver generator.

It is realized by the \rtx\ EIP team of year 2012 and comes in the continuity
with the work done by the year 2009 team.

This document is designed for \rtx\ driver coders and for the EIP laboratory
team.

\end{abstract}

\rtxmaketitleblock

\tableofcontents

\chapter{Prerequisites}

\section{Hardware}

X86 ? 64 bit ?

\section{Software}

Code Worker is needed. You can download it from the official website:
\url{http://codeworker.free.fr/}.

To install the latest version of \rtx\ you need to use Mercurial, the version
control software.

You can download it and find its installation instructions on the following
website: \url{http://mercurial.selenic.com/}.

\chapter{Install \rtx}

\section{Download \rtx}

To download the last build of \rtx. Open a terminal and type:

\begin{verbatim}
hg clone https://rathaxes.googlecode.com/hg/ rathaxes
\end{verbatim}

\section{Compiling \rtx}

\chapter{Driver development}

\section{Intro}

The chapter will learnt you how to develop driver in \rtx.


\section{RTX Language}
%---Conditions-------------------------------

\subsection{Condition}


\subsubsection{Conditional Operator}

Simple condition :

\begin{lstlisting}
ON expression LET sequence(param);
\end{lstlisting}


\subsubsection{Logical Operator}
\begin{lstlisting}
ON expression1 OR expression2 LET sequence1(param)
ON expression3 AND expression4 LET sequence2(param)
ON expression5 AND NOT expression6 LET sequence3(param)
ELSE sequence4(param);
\end{lstlisting}

\subsubsection{Loop}
\begin{lstlisting}
FOREACH TYPE e IN collection {
	// do something
};
\end{lstlisting}


%---Types---------------------------------
\subsection{Types}

\subsubsection{Register}
Syntax :
\begin{lstlisting}
DECLARE REGISTER(RW) BIT[8] Register_Type LIKE(........);
\end{lstlisting}

Instance :
\begin{lstlisting}
DECLARE REGISTER Register_type Register_name Register_type MAPPED_AT
address ;
\end{lstlisting}

Example :
\begin{lstlisting}
DECLARE REGISTER SomeE1000Register eerd PCI MAPPED_AT PCI_BAR_1 +
some_offset ;
\end{lstlisting}


\subsubsection{Primary types}

\begin{lstlisting}
DECLARE
DECLARE
DECLARE
DECLARE
INTEGER i_name = 123;
DOUBLE d_name = 123.123;
BOOLEAN b = TRUE OR FALSE ;
STRING str = `` hello , world ! `` ;
\end{lstlisting}

\subsubsection{Adress}
\begin{lstlisting}
DECLARE ADDRESS addr1 ALIGNED_ON nbrbits OFFSET off ;
\end{lstlisting}

\subsubsection{Buffer}

Example :
\begin{lstlisting}
DECLARE BUFFER MyBufferType SIZE 2048 ADDRESS_ALIGNED_ON 16;
DECLARE MyBufferType mybuffer ;
SET ( my_buffer [3] , unoctet ) ;
\end{lstlisting}


\begin{lstlisting}
DECLARE BUFFER MyBufferType SIZE 2048 ADDRESS_ALIGNED_ON 16 {
  [0..4] AS low_address ;
  [5] AS status {
    (0) -> OK ;
    (1) -> ERROR ;
  };
};
DECLARE MyBufferType mybuffer ;
SET ( my_buffer . status , MyBufferType . status . OK ) ;
\end{lstlisting}

\subsubsection{Collections}

\begin{lstlisting}
DECLARE VECTOR VectorType ELEMENT_TYPE MyBufferType SIZE 10;
DECLARE VectorType my_vector ;
PUSH ( my_vector , element ) ;
POP ( my_vector ) ;
REALLOC_VECTOR ( my_vector , new_size ) ;
\end{lstlisting}

%---Algorithm----------------------
\subsubsection{Algorithm}

\subsubsection{Affectation}

\begin{lstlisting}
DECLARE INTEGER i = 2; // Premiere initialisation
i = 3; // ERROR
SET (i , 3) ; // OK
// Le comportement dans ce cas reste a determiner :
DECLARE VECTOR vtype .....;
DECLARE vtype v ;
SET (v , ...) ; // ERREUR ?
\end{lstlisting}

\subsubsection{Waiting}

\subsubsection{POST and PRE condition}

\subsection{Hello World}

\chapter{Kernel developer}

\section {BLT}

%\rtxbibliography

\end{document}
