\documentclass{rtxreport}

%\usepackage{lipsum}
\usepackage[utf8]{inputenc}

\title{Documentation Utilisateur}
\author{Thomas Luquet}


\rtxdoctype{Documentation utilisateur}
\rtxdocstatus{Brouillon}
\rtxdocversion{0.1}

\rtxdochistory
{
0.1 & 11/03/2011 & Thomas Luquet & start\\
%\hline
%0.2 & 11/09/2010 & Thomas Sanchez & Test with a longer name \\
}

\begin{document}

\maketitle

\begin{abstract}
\rtx 


This document is the user documentation for the \emph{Rathaxes 2012} project. It
includes guide and exemple to install and use Rathaxes. It contain also a description of the language.

\rtx project is a multi-platform peripheral driver generator.

It is realized by the Rathaxes EIP team of year 2012 and comes in the
continuity with the work done by the year 2009 team.

This document is designed for Rathaxes driver coders and for the EIP laboratory
team.

\end{abstract}

\rtxmaketitleblock

\tableofcontents

\chapter{Hardware and software needed}

\section{Hardware needed}

X86 ? 64 bit ?

\section{Software needed}

Code Worker is needed. You can download it on the official website :
http://codeworker.free.fr/

To install the last version of Rathaxes you need to use Mercurial.
Mercurial is a version control software.
You can download and instruction to install it on the official web site :
http://mercurial.selenic.com/


\chapter{Install Rathaxes}

\section{Downloading last version of Rathaxes}
To download the last build of rathaxes. Open a terminal and type :
\begin{verbatim}
hg clone https://rathaxes.googlecode.com/hg/ rathaxes 
\end{verbatim}

\section{Compiling Rathaxes}
\lipsum[3]

\chapter{Driver development}

\section{Intro}
The chapter will learnt you how to develop  driver in Rathaxes.


\section{RTX Language}
%---Conditions-------------------------------

\subsection{Condition}


\subsubsection{Conditional Operator}

Simple condition :

\begin{lstlisting}
ON expression LET sequence(param);
\end{lstlisting}


\subsubsection{Logical Operator}
\begin{lstlisting}
ON expression1 OR expression2 LET sequence1(param)
ON expression3 AND expression4 LET sequence2(param)
ON expression5 AND NOT expression6 LET sequence3(param)
ELSE sequence4(param);
\end{lstlisting}

\subsubsection{Loop}
\begin{lstlisting}
FOREACH TYPE e IN collection {
	// do something
};
\end{lstlisting}


%---Types---------------------------------
\subsection{Types}

\subsubsection{Register}
Syntax :
\begin{lstlisting}
DECLARE REGISTER(RW) BIT[8] Register_Type LIKE(........);
\end{lstlisting}

Instance :
\begin{lstlisting}
DECLARE REGISTER Register_type Register_name Register_type MAPPED_AT
address ;
\end{lstlisting}

Example :
\begin{lstlisting}
DECLARE REGISTER SomeE1000Register eerd PCI MAPPED_AT PCI_BAR_1 +
some_offset ;
\end{lstlisting}


\subsubsection{Primary types}

\begin{lstlisting}
DECLARE
DECLARE
DECLARE
DECLARE
INTEGER i_name = 123;
DOUBLE d_name = 123.123;
BOOLEAN b = TRUE OR FALSE ;
STRING str = `` hello , world ! `` ;
\end{lstlisting}

\subsubsection{Adress}
\begin{lstlisting}
DECLARE ADDRESS addr1 ALIGNED_ON nbrbits OFFSET off ;
\end{lstlisting}

\subsubsection{Buffer}

Example :
\begin{lstlisting}
DECLARE BUFFER MyBufferType SIZE 2048 ADDRESS_ALIGNED_ON 16;
DECLARE MyBufferType mybuffer ;
SET ( my_buffer [3] , unoctet ) ;
\end{lstlisting}


\begin{lstlisting}
DECLARE BUFFER MyBufferType SIZE 2048 ADDRESS_ALIGNED_ON 16 {
  [0..4] AS low_address ;
  [5] AS status {
    (0) -> OK ;
    (1) -> ERROR ;
  };
};
DECLARE MyBufferType mybuffer ;
SET ( my_buffer . status , MyBufferType . status . OK ) ;
\end{lstlisting}

\subsubsection{Collections}

\begin{lstlisting}
DECLARE VECTOR VectorType ELEMENT_TYPE MyBufferType SIZE 10;
DECLARE VectorType my_vector ;
PUSH ( my_vector , element ) ;
POP ( my_vector ) ;
REALLOC_VECTOR ( my_vector , new_size ) ;
\end{lstlisting}

%---Algorithm----------------------
\subsubsection{Algorithm}

\subsubsection{Affectation}

\begin{lstlisting}
DECLARE INTEGER i = 2; // Premiere initialisation
i = 3; // ERROR
SET (i , 3) ; // OK
// Le comportement dans ce cas reste a determiner :
DECLARE VECTOR vtype .....;
DECLARE vtype v ;
SET (v , ...) ; // ERREUR ?
\end{lstlisting}


\subsubsection{Waiting}
\subsubsection{POST and PRE condition}

\subsection{Hello World}

\chapter{Kernel developer}
\subsection {BLT}

\rtxbibliography

\end{document}
