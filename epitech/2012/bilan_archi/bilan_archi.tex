\documentclass{rtxreport}

\author{Thomas Luquet}

\title{Bilan Architecture}

\begin{document}

\maketitle

\rtxmaketitleblock

\tableofcontents

\chapter{Rappel du projet}

\section{Qu'est ce que Rathaxes ?} 
Rathaxes est une suite de logiciel servant à simplifier la création de pilotes de périférique.

\section{Structuration générale de Rathaxes}
Rathaxes est un logiciel fonctionnant en 3 parties :

* Le DSL
* La BlackLibrary
* CodeWorker


\chapter{Diagrame Globale}

Test.

\chapter{Technologie}

\section{Rathaxes, un héritage}
Le choix des technologies à été, en grande parti, choisi par l'équipe 2009 de Rathaxes.
Cependant notre rôle dans le devloppement de l'application sera de comprendre et de redevlopper ces technologies.

\section{Le RTX, un Meta-Langage en construction}

L'un des objectifs du projet rathaxes est d'inclure les technologies DMA et IRQ.
Cest technologies, utilisent des fonctionnalité dite asynchrone

Commande existante
* xdfg
* sth
* sfdth

Commande à ecrire

* xxx
* xxx

\section{La black Library}


\end{document}
