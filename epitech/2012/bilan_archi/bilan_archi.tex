\documentclass{rtxreport}

\author{Thomas Luquet}

\title{Bilan Architecture}

\rtxdoctype{Bilan Architecture}
\rtxdocref{2012\_AA1\_FR\_Rathaxes}
\rtxdocversion{0.5}
\rtxdocstatus{Draft}

\rtxdochistory{
0.1 & 16/11/2010 & Thomas Luquet & Premier draft \\
\hline
0.2 & 17/11/2010 & Louis Opter & Fix compilation \\
\hline
0.3 & 17/11/2010 & Louis Opter & Ajout du cartouche \\
\hline
0.4 & 17/11/2010 & David Pineau & Completion de la section Technologie \\
\hline
0.5 & 17/11/2010 & Louis Opter & Reformulations et ajout de notes \\
}

\newcommand{\note}[1]{\marginpar{\scriptsize{\textdagger\ #1}}}

\begin{document}

\maketitle

\rtxmaketitleblock

\tableofcontents

\chapter{sujet (temp)}

\section{Barème}

Format du document (première page, cartouches, tableau des révisions, résumé,
table des matières, clarté, organisation) /2

Grammaire Orthographe /2

Rappel du fonctionnel de l'application /1

Digramme(s) global de l'application /2

Diagramme(s) détaillé(s) /3

Diagramme(s) de communication(s) (échange de flux, interactions internes des
composants, interactions externes, n'hésitez pas à modéliser via les différentes
représentations UML à votre disposition) /3

Modélisation conceptuelle (même remarque que le point d'avant) /3

Modélisation conceptuelle / physique de la DB /2

Choix des technologies (avantages / inconvénients, étude de pérennité, …) /2

Malus global /-20

\chapter{Rappel du projet}

\section{Qu'est ce que \rtx\ ?}

\rtx\ est un ensemble d'outils permettant de simplifier l'écriture de pilote de
périphérique. Le projet permet de générer un code source écrit en C pour Linux,
Windows 7 et openBSD à partir d'un seul fichier de description de pilote.

C'est un projet à visée scientifique distribué sous licences libres.

Le projet \rtx\ 2012 est une amélioration de l'EIP réalisé en 2009. Il
incorpore de nouvelles fonctionnalitées comme l’asynchronicité. \rtx\ 2012 sera
capable de générer un pilote de souris USB et celui d'une carte son.
% Vous voulez rajouter "qui serviront à prouver que le concept fonctionne"?

\section{Structure du projet}

\rtx\ est divisé en trois parties :

\begin{enumerate}
\item Le langage \rtx\ : un langage dédié (DSL) utilisé pour décrire un pilote~;
\item La black-librarie : Elle permet l'interfaçage entre le DSL et le
compilateur~;
\item Le compilateur : Il transforme les fichiers rathaxes (.rtx) ---à l'aide
de la black-librairie--- en fichier .c spécifiques au système d'exploitation choisi.
\end{enumerate}

\chapter{Diagramme}

\section{Diagramme global}

Diagrame à insérer (et a dessiner)
=>> voir EIP 2009

\section{Diagramme détaillé}

(Obligatoire selon le barem mais je ne vois pas ce qu'on vas pouvoir
mettre dedans)

\chapter{La technologie \rtx}

\section{Un héritage}

Un grand nombre de choix technologiques ont été effectués par l'équipe 2009.
Ils avaient besoin d'une technologie permettant d'implémenter facilement un
compilateur, qui puisse utiliser des modèles de code.

\subsection{Un compilateur}

Pour écrire le compilateur, l'outil retenu par la première équipe est
CodeWorker. CodeWorker est un langage interprété dont la syntaxe s'inspire la
notation EBNF. L'interpréteur CodeWorker est distribué gratuitement sous la
licence libre LGPL.\note{REFERENCE NEEDED}

CodeWorker, présente l'avantage d'être facilement abordable et permet une
analyse syntaxique et lexicale avancée, tout en permettant la génération de
code en se basant sur l'arbre de syntaxe abstraite généré par le code
précédemment analysé\note{EPHRASETOOLONG}.

De plus, la syntaxe proposée pour le langage de script de CodeWorker est
pour\note{c'est moi ou ce paragraphe veut rien dire ?} la
partie analyse syntaxique relativement proche de la syntaxe Backus-Naur Form,
sous sa version étendue; et la syntaxe proposée pour le script de manipulation
d'arbre syntaxique et de génération est dérivée du C, la rendant relativement
familière a l'ensemble de l'équipe qui maitrise le C.

\subsection{Une série de modèles}

\note{METTRE AU PRÉSENT}

L'objectif du projet était de générer des drivers en C pour différents systèmes
d'exploitation. Ceci devait être fait en abstrayant les problématiques de
connaissances spécifiques à un OS. Il était donc nécessaire de posséder un
système de modèles de code utilisables pour générer du code C sous un système
spécifique.

C'est pour cela qu'une fois de plus CodeWorker fut choisi comme l'outil le plus
pratique. En effet, la solution la plus simple pour avoir des modèles de code
qui permettaient de générer du code C était d'écrire directement des modèles
en script CodeWorker, de manière a réduire la quantité de travail nécessaire
du côté du backend de \rtx\ (pour la partie parsing).

\section{Un langage}

Forts de l'expérience acquise par l'équipe 2009 de \rtx, l'objectif de
notre groupe est de rechercher et implémenter de nouveaux concepts dans
\rtx. Parmi ceux-ci, s'illustre notemment les notions d'asynchronicité,
de DMA (Direct Memory Access), d'IRQ (Interrupt ReQuest) et enfin l'ensemble
des concepts liés a l'utilisation de BUS tels que le PCI dans les pilotes de
périphériques.

Un autre objectif de notre équipe est d'apporter des améliorations dans le
fonctionnement même de \rtx. Jusqu'alors, le code C était généré par
activation des modèles en fonction du contenu de l'arbre de syntaxe abstraite.
Nous désirons changer cela, pour d'une part rendre le coeur du compilateur
plus générique, et permettre de tester efficacement chaque étape
d'implémentation du langage.

Pour cela, nous allons conserver le CodeWorker, qui par sa simplicite, nous
permettra de modifier le coeur du compilateur, afin par la suite de pouvoir
facilement intégrer de nouveaux concepts et mots clefs au compilateur,
ainsi que d'intégrer plus aisément les contributeurs externes au projet.

\end{document}
