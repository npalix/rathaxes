\documentclass{rtxreport}

\author{Thomas Luquet}

\title{Bilan Architecture}

\rtxdoctype{Bilan Architecture}
\rtxdocref{2012\_AA1\_FR\_Rathaxes}
\rtxdocversion{0.3}
\rtxdocstatus{Draft}

\rtxdochistory{
0.1 & 16/11/2010 & Thomas Luquet & Premier draft \\
\hline
0.2 & 17/11/2010 & Louis Opter & Fix compilation \\
\hline
0.3 & 17/11/2010 & Louis Opter & Ajout du cartouche \\
\hline
0.4 & 17/11/2010 & David Pineau & Completion de la section Technologie \\
}

\begin{document}

\maketitle

\rtxmaketitleblock

\tableofcontents

\chapter{sujet (temp)}

\section{Barème}

Format du document (première page, cartouches, tableau des révisions, résumé,
table des matières, clarté, organisation) /2\\
Grammaire Orthographe /2\\
Rappel du fonctionnel de l'application /1\\
Digramme(s) global de l'application /2\\
Diagramme(s) détaillé(s) /3\\
Diagramme(s) de communication(s) (échange de flux, interactions internes des
composants, interactions externes, n'hésitez pas à modéliser via les différentes
représentations UML à votre disposition) /3\\
Modélisation conceptuelle (même remarque que le point d'avant) /3\\
Modélisation conceptuelle / physique de la DB /2\\
Choix des technologies (avantages / inconvénients, étude de pérennité, …) /2\\
Malus global /-20\\

\chapter{Rappel du projet}

\section{Qu'est ce que Rathaxes ?} 

Rathaxes est un ensemble d'outils permettant de simplifier l'écriture de pilote
de périphérique. Le projet permet de générer un code source écrit en C pour
Linux, Windows 7 et openBSD a partir d'un seul fichier de description de pilote.
Ses principaux outils sont :

Le projet Rathaxes 2012 est une amélioration d'un EIP réalisé en 2009. Il
incorpore de nouvelles fonctionnalités comme l’asynchronicité. Rathaxes 2012
sera capable de générer un pilote de souris USB et celui d'une carte son en
preuve de concepte.  C'est un projet à visée scientifique distribué sous
licences libres.


\section{Structuration générale de Rathaxes}

Rathaxes est fonctionne avec3 parties :

\begin{itemize}
        \item Le langages Rathaxes : C'est un langage dédié (DSL),
            il est utilisé pour décrire un pilote.
        \item La black-librairie :  Elle permet l'interfacage entre le DSL
            et le compilateur;
        \item Le compilateur : Il transforme les fichiers rathaxes (.rtx),
            à l'aide de la black-librairie, en fichier .c spécifique à la
            plate-forme. \ldots
\end{itemize}

\chapter{Diagramme}

\section{Diagramme global}

Diagrame à insérer (et a dessiner) 
=>> voir EIP 2009


\section{Diagramme détaillé}

(Obligatoire selon le barem mais je ne vois pas ce qu'on vas pouvoir
mettre dedans)

\chapter{Technologie}

\section{Rathaxes, un héritage}

Un grand nombre de choix sur les technologies ont été effectués par
l'équipe 2009 de Rathaxes. Ils avaient en effet pour ce projet le besoin d'une
technologie permettant d'implementer facilement un compilateur, qui puisse
utiliser des modèles de code.

\subsection{Le Compilateur}

Pour écrire le compilateur, l'outil retenu par la première équipe de Rathaxes
fut le CodeWorker, un langage interpreté accompagné par un executable
(l'interpreteur), distribué gratuitement sous licence LGPL (GNU Lesser General
Public Licence).
\\
\\
Cet outil présente l'avantage d'être facilement abordable, et de permettre
une analyse syntaxique et lexicale avancée, tout en permettant la génération
de code en se basant sur l'arbre de syntaxe abstraite généré par le code
précédemment analysé. Son avantage sur les autres outils de type similaire tels
que
De plus, la syntaxe proposée pour le langage de script de CodeWorker est pour
la partie analyse syntaxique relativement proche de la syntaxe Backus-Naur Form,
sous sa version étendue; et la syntaxe proposée pour le script de manipulation
d'arbre syntaxique et de génération est dérivée du C, la rendant relativement
familière a l'ensemble de l'équipe qui maitrise le C.

\subsection{Le Backend : une série de modèles}

L'objectif du projet était de générer des drivers en C pour différents systèmes
d'exploitation. Ceci devait être fait en abstrayant les problématiques de
connaissances spécifiques à un OS. Il était donc nécessaire de posséder un
système de modèles de code utilisables pour générer du code C sous un système
spécifique.
\\
\\
C'est pour cela qu'une fois de plus CodeWorker fut choisi comme l'outil le plus
pratique. En effet, la solution la plus simple pour avoir des modèles de code
qui permettaient de générer du code C était d'écrire directement des modèles
en script CodeWorker, de manière a réduire la quantité de travail nécessaire
du côté du backend de Rathaxes (pour la partie parsing).


\section{Rathaxes, un Meta-Langage en construction}

Forts de l'expérience acquise par l'équipe 2009 de Rathaxes, l'objectif de
notre groupe est de rechercher et implémenter de nouveaux concepts dans
Rathaxes. Parmi ceux-ci, s'illustre notemment les notions d'asynchronicité,
de DMA (Direct Memory Access), d'IRQ (Interrupt Request) et enfin l'ensemble
des concepts liés a l'utilisation de BUS tels que le PCI dans les pilotes de
périphériques.
\\
Un autre objectif de notre équipe est d'apporter des améliorations dans le
fonctionnement même de Rathaxes. Jusqu'alors, le code C était généré par
activation des modèles en fonction du contenu de l'arbre de syntaxe abstraite.
Nous désirons changer cela, pour d'une part rendre le coeur du compilateur
plus générique, et permettre de tester efficacement chaque étape
d'implémentation du langage.
\\
\\
Pour cela, nous allons conserver le CodeWorker, qui par sa simplicite, nous
permettra de modifier le coeur du compilateur, afin par la suite de pouvoir
facilement intégrer de nouveaux concepts et mots clefs au compilateur,
ainsi que d'intégrer plus aisément les contributeurs externes au projet.


\end{document}
