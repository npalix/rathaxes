\documentclass{rtxreport}

\author{Thomas Luquet}

\title{Devis Conférence RMLL}

\rtxdoctype{Devis Voyage}
\rtxdocref{2012\_DEVIS\_RMLL\_FR\_RATHAXES}
\rtxdocversion{0.2}
\rtxdocstatus{Brouillon}

\rtxdochistory{
0.1 & 16/11/2010 & Thomas Luquet & Premier draft \\
\hline
0.2 & 14/05/2011 & Louis Opter & Fixe compilation \\
}

\begin{document}

\maketitle

\rtxmaketitleblock

\tableofcontents

\chapter{Rappel du projet}

\section{Qu'est ce que \rtx\ ?}

\rtx\ est un ensemble d'outils permettant de simplifier l'écriture de pilote de
périphérique. Le projet permet de générer un code source écrit en C pour Linux,
Windows 7 et OpenBSD à partir d'un fichier de description de pilote.

Le projet \rtx\ 2012 est une amélioration de l'EIP réalisé en 2009. Il
incorpore de nouvelles fonctionnalitées comme l’asynchronicité. \rtx\ 2012 sera
capable de générer un pilote de souris USB et celui d'une carte son qui
serviront à prouver que le concept fonctionne.

Nous ciblons la communauté scientifique et nous avons décidé de rendre le
développement du projet publique grâce à un dépôt Google
code\footnote{\url{http://code.google.com/p/rathaxes/}} et des licences libres.



\chapter{Qu'est ce que les RMLL ?}\footnote{ Source wikipedia \url{http://fr.wikipedia.org/wiki/RMLL}}

\section{Présentation des Rencontres Mondiale du Logiciel Libre}


Les Rencontres mondiales du logiciel libre se déroulent chaque année, depuis 2000, au début du mois de juillet. Créées à l'initiative de l’Association bordelaise des utilisateurs de logiciels libres (ABUL), elles ont eu lieu plusieurs années à Bordeaux, puis dans d'autres villes françaises. La fréquentation de l'événement augmente régulièrement : alors que la première édition avait rassemblé plus de 1 000 personnes venues de plus de 50 pays des cinq continents1, ce sont plus de 4 000 visiteurs qui se sont déplacés pour assister à l'édition 20082.

Les RMLL sont aujourd'hui une combinaison de quatre manifestations simultanées complémentaires :
\begin{enumerate}
    \item une rencontre de concepteurs et développeurs venant échanger sur leurs projets dans une ambiance conviviale ;
    \item une manifestation de sensibilisation au logiciel libre s'adressant à un large public ;
    \item un rendez-vous des bénévoles avec le « Village des Associations » ;
    \item un lieu de formation personnelle et professionnelle.
\end{enumerate}
L'organisation de conférences techniques est pour les spécialistes l'occasion d'échanger sur leurs projets, tandis que l'organisation de cycles de conférences « grand public » et d'ateliers pratiques permet à tous de découvrir les logiciels libres et de dialoguer avec les auteurs et utilisateurs des logiciels.

Afin de toucher le plus large public possible, les RMLL sont en accès libre et gratuit, et s'appuient sur des structures locales (universités, IUFM, CROUS, chambres de commerce, etc.) pour fournir des prestations d'accueil des conférences, d'hébergement et de restauration à coût très modique. Le « Repas du Libre » est le principal événement social de la manifestation ; 413 convives ont participé à celui de l'édition 2008 à Mont-de-Marsan2.

\section{Organisation}

Les RMLL sont constituées, d'une part, de fils de conférences thématiques, d'autre part, de stands où le visiteur peut prendre contact avec des associations ou des entreprises. Les thèmes vont du développement de noyaux et de systèmes d'exploitation à l'étude du contexte légal, économique et politique du logiciel libre en passant par l'éducation et l'éducation populaire. Si les aspects techniques restent un thème central dont la qualité est à la base du renom des RMLL, le champ de la manifestation s'est élargi d'année en année et intéresse la culture, l'art, les loisirs, l'accessibilité et le handicap, l'économie sociale et solidaire, le développement durable, la solidarité internationale…

Les RMLL sont organisées par une équipe de bénévoles avec le soutien matériel et financier d'organismes publics, privés et de collectivités territoriales partenaires (régions, départements, communautés urbaines ou d'agglomération, communes) qui aident l'association locale responsable.

Des ''fils rouges'' de découverte du Libre, sélection transversale aux thèmes, sont proposés selon les publics. En plus des conférences et tables rondes, de nombreux ateliers sont proposés.

\section{Village}

Le visiteur des RMLL est accueilli par un village associatif, ensemble de stands permettant aux associations locales et nationales de présenter leurs activités et leur projet. Un village des entreprises est également organisé en collaboration avec la Chambre de commerce et d'industrie. Ces villages sont une vitrine du dynamisme des acteurs, commerciaux ou non, du logiciel libre. Ils rendent possible une prise de contact informelle et conviviale.


\chapter{Objectifs du voyage}
% Ou pourquoi y allé
Nous souhaitons participer aux RMLL pour plusieurs raisons :

Tout d'abord nous souhaitons mettre en place une communauté d'utilisateurs de Rathaxes. Faire une conférence nous semble un moyen simple et efficace de communiquer sur le projet.

Ensuite, Rathaxes est un projet open-source, nous profiterons de cette conférence pour trouver de nouveaux contributeurs au projet, pour participer au développement ainsi qu’à l’intégration de nouveaux concepts.
Déjà, lors de la dernière conférence qu'avait fait l'équipe de 2009 (RMLL 2008), des contacts avaient été pris. Il s'agit de confirmer ces contacts et de montrer l'évolution du projet.
Par ailleurs, de cette rencontre avec d'autres développeurs, nous espérons recevoir des idées critiques sur Rathaxes afin d'améliorer continuellement la qualité du projet.

Le projet étant basé sur la thèse du Docteur Laurent Réveillère, il nous semble important d'exposer les avancées scientifiques qui ont été réalisées grâce à Rathaxes.

Enfin, cette conférence sera l'occasion de faire la promotion de l'école.


\section{Conférence accpeté}

Notre demande de conférence à été accepté par les organisateurs du RMLL. 
Cette conférence devant un publique d'expert sera pour nous l'occasion de travailler notre orale et nos présentations. Ceci nous semble allé dans l'esprit pédagogique des soutences de {EPITECH}.


\section{Communoté}
% projet open-source -> communoté ++
Le projet \{rtx}\ étant un projet OpenSource, nous comptons sur ces présentations pour augmenter la taille de notre communoté. Par ces rencontres, nous pourrons développer notre productivité et recevoir un avis extérieurs sur le projet ce qui améliora la qualité de notre code.



\section{Stande Rathaxes}
Nous avons fait une demande de stande pour le ''Village RMLL'' pour les RMLL.
Nous attendons maintenant la réponse.

\section{Recherche Scientifique}

\chapter{Intéret pour l'école}
%bulshit !
Lors de la conférence et sur le stande nous ferons la promotion de l'école et du système EIP qui nous à permis de faire exister le projet {\rtx\}

\section{Conférence filmé et diffuser sur internet}
Notre conférence sera filmé et diffuser en directe sur internet. Par aillieur l'enssemble de la blogosphère technophile couvrira l'évenment. C'est donc un avantage de communication externe non négligable pour notre cible commerciale.

\section{Article pour le site de l'école} 

Nous avons rencontrer le service communication de l'école qui nous à fait par de son intéret sur l'evenement. Ils souhaitent que nous prenions des photos et que nous réalision une mini-interview sur place. Ce que nous ferons avec intéret.

\chapter{Devis}

\section{Train}

\begin{tabular}{|l|l|c|r|r}
  \hline
  Login &  Carte 12-25 & Aller 2 & Retour & Totale \\
  \hline
  pineau_d & Oui & 23.9 & 23.9 & 47.8 \\
  konarz_z & Non & 25 & 25 & 50 \\
  sanchez_t & Non & 25 & 25 & 50 \\
  opter_l & Oui & 23.9 & 23.9 & 47.8 \\
  luquet_t & Oui & 23.9 & 23.9 & 47.8 \\
  Lionel Auroux & Non & 25 & 25 & 50  \\
  Total & & & & 293.4 \\
  \hline

\end{tabular}

Notons que le prix du train varie en fonction du jour où nous le réservons. Aussi pour plus de sécurité nous majorerons le prix du train de 20 \%.
Ce qui donne un prix de {\bf 322 Euros}.


\section{Hotel}

Un hôtel proche du centre ville à été choisi afin de nous permettre d'accéder aux centres de conférence facilement et cela sans utiliser de voiture.
Pour 6 personnes la formule hôtel + petit-déjeuner est estimé à 378 Euros.


\subsection{Flyers et T-Shirt}

% Ne pas dire Rathaxes ici car juste en dessous on dit que c'est un
% compilateur…

%\begin{enumerate}
%\item Le langage \rtx\ : un langage dédié (DSL\footnote{Domain Specific
%Language.}) utilisé pour décrire un pilote~;
%\item La \BL: Elle permet l'interfaçage entre le DSL et le compilateur~;
%\item Le compilateur : Il transforme les fichiers \rtx\ (\texttt{.rtx}) ---à
%l'aide de la \BL--- en fichier \texttt{.c} spécifiques au système d'exploitation
%choisi.
%\end{enumerate}


\end{document}
