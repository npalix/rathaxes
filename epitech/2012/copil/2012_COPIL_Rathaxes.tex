\documentclass[francais]{rtxreport}

\rtxdoctype{COPIL}
\rtxdocstatus{Brouillon}
\rtxdocversion{0.4}

\rtxdochistory
{
0.1 & 15/05/2011 & Louis Opter & Création du document à partir du template .docx \\
\hline
0.2 & 15/05/2011 & Louis Opter & Rédaction des chapitres sur la documentation et la méthodologie \\
\hline
0.3 & 16/05/2011 & Thomas Sanchez & Rédaction du chapitre sur la réalisation technique \\
\hline
0.4 & 16/05/2011 & Louis Opter & Relecture \\
}

\author{Louis Opter}
\title{COPIL -- Comité de Pilotage}

\begin{document}

\maketitle

\begin{abstract}
\rtx\ est un générateur, multi-plateformes, de pilotes de périphériques
matériels. Le fonctionnement du périphérique est décrit dans le langage
spécialisé \rtx\ avant d'être compilé vers un module noyau écrit en C pour
Linux, Windows ou OpenBSD.

Ce document synthétise l'avancée du projet depuis le démarrage de l'EIP. Il
présentera rapidement l’avancée technique et documentaire pour finir sur les
actions de communication prévues.
\end{abstract}

\rtxmaketitleblock

\tableofcontents

\chapter{Présentation du projet}

\emph{Ce chapitre doit faire \textbf{une page}.}

\section{Cadre du projet}

EIP - Epitech.

\section{Contexte}

LSE.

\chapter{Étude de l'existant}

\emph{Ce chapitre doit faire \textbf{une page}.}

\section{État de l'art}

\section{Positionnement par rapport à l'existant}

% Aucune idée de quoi il s'agit

\chapter{Réalisation technique}

\emph{Ce chapitre doit faire \textbf{deux pages}.}

\rtx\ est un projet difficile qui requiert beaucoup de recherche et de
validation de résultats intermédiaires. Lorsque \rtx\ a été écrit pour la
première fois, certains aspects du matériel ont été laissés de côté et donc il
ne gère donc que certains types de cartes maintenant obsolètes. Aujourd'hui,
nous avons pu réécrire les bases de \rtx\ prenant en compte les résultats
dégagés par nos recherches et sommes capables de manipuler des arbres de
syntaxes de manière souple, efficace et maintenable. Depuis cette réécriture
nous sommes capables de valider point par point chaque étape de la génération du
pilote final: cela nous permet de valider nos ajouts de fonctionnalité et éviter
les régressions.

Nous estimons à 70\% la progression de l’écriture du compilateur. Niveau
fonctionnalités pures du compilateur, nous en sommes au niveau de l'équipe 2009.

Il nous reste à écrire les interfaces systèmes (c.-à-d. les fonctions systèmes
appelées réellement) pour chaque système d’exploitation.

\section{Architecture}
\rtx\ se divise en trois parties:
\begin{enumerate}
	\item Le langage ;
	\item Le compilateur ;
	\item Les bibliothèques.
\end{enumerate}

Le langage est ce que le développeur voit au premier abord : de la même manière
que l’on apprend un langage de programmation, l’utilisateur devra apprendre à
s’exprimer avec \rtx.

Le compilateur permettra de traduire un fichier écrit dans ce langage en pilote
de périphérique. Pour accomplir cette tâche, il utilisera les bibliothèques
fournies avec le logiciel. Les bibliothèques contiennent le «~savoir~» du
compilateur : sans elles, le compilateur pourrait lire les fichiers \rtx, mais
il ne saurait absolument pas quoi générer par la suite.

Le compilateur lui-même se divise en trois parties, le \emph{front},
\emph{back} et \emph{middle} \emph{-end}.

\begin{itemize}
	\item Le \emph{font-end} est la partie construisant l’arbre de syntaxe à partir du
	fichier ;
	\item Le \emph{middle-end} est la partie qui traite l’arbre généré par le
	\emph{front end}, l’instrumente (l’altère avec des informations supplémentaires)
	avec les informations recueillies via le back-end afin de régénérer un nouvel
	arbre qui peut être reconvertis en code ;
        \item Le \emph{back end} est toute la partie faisant l’interface avec le
        code des bibliothèques afin de le manipuler.
\end{itemize}

\section{Avancement}

Bien que l'équipe \rtx\ soit divisée (entre les États-Unis et la France), du
fait de sa petite taille avance plutôt vite. Le retour d'expérience de l'équipe
2009 nous a aidé à ne pas refaire les mêmes erreurs.

Nous prévoyons une preuve de concept avec un pilote d’une carte PCI d’ici l'été
2011 pour les soutenances finales. Il nous est très difficile de nous avancer à
plus long terme, mais nous prévoyons d’être concentrés sur l’écriture des
fichiers de la bibliothèque pendant notre 5\textsuperscript{ème} année.

\section{Problèmes rencontrés et résolus}

L’équipe n’a pas tant rencontré de problèmes que ça. Les «~problèmes~» sont plus
des décisions et orientations techniques à prendre. Il est encore trop tôt pour
dire si elles sont toutes bonnes jusqu’à maintenant.

En revanche, lorsque nous avons dû nous former techniquement et écrire des
pilotes, beaucoup de problèmes sont apparus. Nous pouvons citer : le manque de
documentation, l’étendue des connaissances à acquérir, la complexité des
documentations constructeurs (et leur rareté). De plus, les codes existants sont
rarement commentés, très optimisés et complexes (car ils gèrent souvent
plusieurs cartes à la fois).

De plus, on les connait moins, mais ils existent: les bugs matériels. Ils
rendent encore plus difficile la lecture des codes existants.

\section{Synthèse des bilans techniques}

Nos bilans techniques se sont malheureusement rarement bien passés. Il est vrai
que nous sommes un peu «~dans notre monde~» et on oublie un peu le monde réel.
Par conséquent, il nous est arrivé de passer à côté d’une soutenance en
arrivant avec des documentations manquantes ou ne répondant pas aux exigences
du laboratoire EIP.

Nous avons aussi eu des problèmes de communication vis-à-vis du travail que
nous fournissons.

Gardant en tête tout cela, nous faisons notre possible pour que tout se passe
mieux et que notre travail soit justement mis en valeur afin qu’il puisse être
évalué à sa juste valeur.

Nous espérons que le laboratoire EIP et ses correcteurs pourront noter les
améliorations et nous aideront à surmonter les erreurs restantes.

\chapter{Conduite de projet}

\emph{Ce chapitre doit faire \textbf{deux pages}.}

\section{Organisation du groupe}

\rtx\ est un groupe de cinq étudiants :
\begin{description}
\item[Thomas Luquet] : Thomas s'occupe des relations extérieures (tickets pour
le labEIP, organisation des voyages\ldots) et du planning (rappel des réunions,
des dates de rendus et inscription aux soutenances) ;
\item[Louis Opter] : Louis s'occupe de gérer le dépôt du code, du système de
compilation et de distribution (paquets/installateurs), du site vitrine, de la
mise en forme de tous les documents du projet et de la documentation
utilisateur ;
\item[David Pineau] : David s'occupe de la conception du langage, du
compilateur écrit en CNorm et de la documentation technique du langage et du
compilateur ;
\item[Thomas Sanchez] : Thomas s'occupe de la conception du langage grâce aux
concepts techniques qu'il obtient en développant des pilotes modèles ;
\item[Zoltan Konarzewski] : Zoltan assiste David sur la réalisation du
compilateur.
\end{description}

%Enfin, il convient de citer Lionel Auroux qui ---avec son expérience et sa
%vision sur le long terme--- arbitre les décisions à prendre et les prochains
%points techniques sur lesquels travailler.

\section{Méthodologie}

La méthodologie du groupe s'appuie autour du dépôt de code hébergé sur le
service Google Code. Le dépôt de code est au format Mercurial qui a été retenu
pour deux raisons :
\begin{itemize}
\item Son fonctionnement décentralisé qui facilite l'ajout de nouvelles
fonctionnalités étant donné que chaque développeur possède son propre dépôt ;
\item Sa facilité d'utilisation ---en particulier sous Windows--- et sa facilité
de prise en main par les utilisateurs de Subversion.
\end{itemize}

Le dépôt contient non seulement l'avancée actuelle de \rtx\ mais aussi sa
documentation et nos document de travail. L'utilisation de Mercurial pour tous
nos documents et toutes nos présentation est une volonté méthodologique : le
dépôt est l'unique référence pour tous ce qui concerne le développement de \rtx.
Afin de pouvoir versionner nos documents nous avons font le choix de les écrire
en \LaTeX\ qui a aussi ---comme Mercurial--- l'avantage de fonctionner sur tous
les systèmes d'exploitations.

Google Code nous évite un travail d'administration (nous n'avons pas besoin de
gérer nos propres serveurs) et en plus d'héberger notre dépôt Mercurial de
référence il permet aussi:
\begin{itemize}
\item D'avoir une façade publique ou les développeurs intéressés peuvent
facilement consulter nos documentations et télécharger le projet ;
\item Un répertoire de bogues publique.
\end{itemize}
Google nous permet aussi d'héberger une liste de diffusion facilement, dont nous
nous servons intensivement pour tous nos débats internes.

Le canal IRC \texttt{\#rathaxes} dont nous nous servons pour nos discussions
internes et qui pourra servir d'outils de supports pour nos utilisateurs, est
enregistré sur le réseau freenode.

Enfin, notre système de compilation possède plusieurs automatismes pour
faciliter le développement et la distribution du projet :
\begin{itemize}
\item Exécution automatisé des tests unitaires et d'intégration aussi bien sous
Windows et que les autres plateformes ;
\item Téléchargement automatique des dépendances du projets ;
\item Génération automatique d'un installateur Windows et des archives sources
pour Windows ou les autre plateformes.
\end{itemize}

\section{Organisation des réunions}

Les réunions sont organisées chaque Samedi à 19 heures CEST en visio-conférence
sur Skype. Nos réunions suivent toutes le même planning :
\begin{itemize}
\item Thomas L. fait le tour sur les points administratifs à voir (tickets avec
le labEIP, date butoirs des candidatures aux conférences\ldots) et nous
rappel les prochains événements sur le calendrier du labEIP ;
\item Point sur l'avancement par rapport a notre planning et celui du labEIP ;
\item Discussions sur les décisions technique à prendre (si besoin) ;
\item Distribution du travail.
\end{itemize}

Un résumé de la réunion est rédigé par Thomas L. et envoyé sur notre liste de
diffusion interne.

L'utilisation de moyens de communications électroniques est justifiée par :
\begin{itemize}
\item Le fait que deux des membres du groupes se trouvent en Californie avec
neuf heures de décalage horaire par rapport à la France ;
\item La volonté d'avoir un modèle de développement qui laisse la possibilité à
des développeurs extérieurs à Epitech de pouvoir facilement suivre notre
avancement et contribuer au projet.
\end{itemize}

\chapter{Points sur les documentations}

\emph{Ce chapitre doit faire \textbf{une page}.}

\section{Organisation des documentations}

La documentation du projet comporte :
\begin{itemize}
\item Une documentation interne avec des documents de travail qui servent
d'appuis au développement du langage et du compilateur ;
\item Une documentation utilisateur qui détaille aussi bien l'utilisation de
\rtx\ qu'une référence complète du langage et de l'implémentation du
compilateur.
\end{itemize}

La documentation interne contient des documents de travail : soit des prototypes
du langage ou du compilateur, soit des documents qui expliquent et décomposent
des pilotes de périphériques existant qui nous servent de références lors de nos
réflexions sur les fonctionnalités nécessaires dans le langage.

La documentation utilisateur, traduite en anglais et en français, est elle même
découpées en deux parties :
\begin{itemize}
\item Un «~tutoriel~» qui explique comment installer \rtx\ sous Windows ou un
autre système d'exploitation et qui contient des liens vers les autres
documentations ;
\item Une partie technique qui couvre les aspects \emph{front}, \emph{middle} et
\emph{back~-end} du langage.
\end{itemize}
Ce découpage de la documentation technique a été choisi car elle s'adresse à des
personnes potentiellement différentes avec des compétences différentes :
fabricant de matériel, développeur de système d'exploitations, développeur de
pilotes.

\section{Avancement}

De nouveaux documents de travail seront sans doutes rédigés mais il ne semble
pas nécessaire d'ajouter de nouveaux documents «~utilisateurs~» : la
documentation rédigée jusqu'ici décrit déjà l'ensemble du projet de
son installation à son utilisation. Seul un travail de mise à jour sur les
parties qui sont encore en développement est prévue.

\chapter{Communication}

\emph{Ce chapitre doit faire \textbf{une page}.}

% - Quels cours suivez-vous
% - Ce que cela vous apporte pour adapter votre discours en fonction des interlocuteurs

\chapter{Promotion extérieure}

\emph{Ce chapitre doit faire \textbf{une page}.}

% - Où en êtes-vous
% - Quels suivis faites-vous
% - Apports

\chapter{Conclusion}

\emph{Ce chapitre doit faire \textbf{une page}.}

% - Où en êtes-vous
% - Avez-vous des modifications à faire dans votre CDC ?
% - Quel avenir dans les 6 mois (fin d’année)


\end{document}
