\documentclass[chapterprefix=off,francais]{rtxreport}

\rtxdoctype{COPIL}
\rtxdocstatus{Brouillon}
\rtxdocversion{0.1}

\rtxdochistory
{
0.1 & 11/03/2011 & Louis Opter & Création du document à partir du template .docx \\
}

\author{Louis Opter}
\title{COPIL -- Comité de Pilotage}

\begin{document}

\maketitle

\begin{abstract}
\rtx\ est un générateur, multi-plateformes, de pilotes de périphériques
matériels. Le fonctionnement du périphérique est décrit dans le langage
spécialisé \rtx\ avant d'être compilé vers un module noyau écrit en C pour
Linux, Windows ou OpenBSD.

Ce document synthétise l'avancée du projet depuis le démarrage de l'EIP. Il
présentera rapidement l’avancée technique et documentaire pour finir sur les
actions de communication prévues.
\end{abstract}

\rtxmaketitleblock

\tableofcontents

\chapter{Présentation du projet}

\emph{Ce chapitre doit faire \textbf{une page}.}

\section{Cadre du projet}

EIP - Epitech.

\section{Contexte}

LSE.

\chapter{Étude de l'existant}

\emph{Ce chapitre doit faire \textbf{une page}.}

\section{État de l'art}

\section{Positionnement par rapport à l'existant}

% Aucune idée de quoi il s'agit

\chapter{Réalisation technique}

\emph{Ce chapitre doit faire \textbf{deux pages}.}

\section{Rappel de l'architecture}

\section{Avancement}

\section{Problèmes rencontrés et résolus}

\section{Synthèse des bilans techniques}

\chapter{Conduite de projet}

\emph{Ce chapitre doit faire \textbf{deux pages}.}

\section{Organisation du groupe}

\section{Méthodologie}

\section{Organisation des réunions}

\chapter{Points sur les documentations}

\emph{Ce chapitre doit faire \textbf{une page}.}

\section{Organisation des documentations}

\section{Avancement}

\chapter{Communication}

\emph{Ce chapitre doit faire \textbf{une page}.}

% - Quels cours suivez-vous
% - Ce que cela vous apporte pour adapter votre discours en fonction des interlocuteurs

\chapter{Promotion extérieure}

\emph{Ce chapitre doit faire \textbf{une page}.}

% - Où en êtes-vous
% - Quels suivis faites-vous
% - Apports

\chapter{Conclusion}

\emph{Ce chapitre doit faire \textbf{une page}.}

% - Où en êtes-vous
% - Avez-vous des modifications à faire dans votre CDC ?
% - Quel avenir dans les 6 mois (fin d’année)


\end{document}
