\documentclass[chapterprefix=off]{rtxreport}

\author{Louis Opter}
\title{Cahier des charges pour 2012}

\rtxdoctype{Test Document}
\rtxdocstatus{Brouillon}
\rtxdocversion{0.1}

\rtxdochistory
{
0.1 & 11/03/2011 & Louis Opter & Reprise du CDC1 \\
}

\begin{document}

\maketitle

\rtxmaketitleblock

\tableofcontents

\chapter{Introduction}

\section{\rtx}

% Rajouter un logo avec wrapfig

Ce document est le cahier des charges pour le projet \rtx\  2012. Il comporte
une description du travail à accomplir pour achever ce projet. \rtx\  est un
générateur de pilote de périphériques d'ordinateurs multi-plateformes.

\rtx est réalisé par l'équipe EIP \rtx 2012 et s'inscrit dans la continuité du
travail effectué par l'équipe 2009.

Ce document s'adresse aux développeurs \rtx\  et à l'équipe du laboratoire EIP.
Il décrit le projet \rtx, ses buts, ses moyens et ses limites.

Ce projet est réalisé en partenariat avec le LSE : \emph{«~Laboratoire Système et
Sécurité Epita et Epitech~»}.

\section{Laboratoire EIP}

% Rajouter un logo eip avec wrapfig

EIP est un acronyme pour : \emph{«~Epitech Innovative Project~»}.

Les EIP désignent les projets de fin d'études réalisés par les étudiants
d'Epitech à partir de leur troisième année.

L'équipe pédagogique du laboratoire EIP prend en charge les étudiants sur toute
la durée de leurs EIP avec des suivis, des corrections et une infrastructure.

\chapter{\rtx\ 2009}

\section{Problématique}

Écrire un pilote matériel requiert des connaissances approfondies sur comment le
matériel et le système fonctionnent. Les pilotes sont lancés avec un haut niveau
de privilège et peuvent causer des désastres si quelque chose est mal fait
(tel qu'un arrêt brutal de la machine).

Chaque plateforme propose ses propres interfaces de communication et les
pilotes doivent être écrits pour chacune d'elles.

Au final, il semble évident que des logiciels sont manquants pour contrer ces
problèmes :
\begin{itemize}
\item Séparation entre les compétences matériel et logicielle ;
\item Temps de développement ;
\item Réutilisation du code.
\end{itemize}

\section{Vue d'ensemble du projet}

Le projet est divisé en quatre parties :
\begin{enumerate}
\item Le DSL\footnote{\emph{Domain Specific Language}} qui décrit un pilote
périphérique.
\item La \BL\ est une bibliothèque de patrons utilisée lors de la génération
du pilote.
\item Un compilateur qui traduit le DSL \rtx\  et génère un pilote.
\item Une documentation abondante sur le langage et la \BL\ pour les
utilisateurs et contributeurs de \rtx.
\end{enumerate}

Le DSL et le compilateur sont distribués sous licence GPLv3\cite{GPL3} et
la \BL\ sous licence BSD\cite{BSD}.

Actuellement, \rtx\  est une preuve de concept qui peut générer un pilote RS-232.

\chapter{Objectifs pour 2012}

À la fin de la session EIP 2012, le projet devra être utilisable par n'importe
quel programmeur de pilote avec des exemples sur du matériel actuel.

Le projet suivra plusieurs étapes qui comportent :
\begin{enumerate}
\item Des recherches sur la sémantique des bus (comme PCI, USB, i2c, \ldots) ;
\item Implémentation des bus et de leurs algorithmes dans le DSL ;
\item Ajouter de nouveaux patrons de code dans la \BL.
\end{enumerate}

Tout au long du projet, l'équipe \rtx\  sera montrera active dans la communauté
Open Source.

\chapter{Contraintes}

\section{Le DSL}

Le DSL \rtx\ doit :
\begin{itemize}
\item Décrire simplement et efficacement un pilote périphérique ;
\item Être capable d'intégrer des morceaux de code C ;
\item Être naturel au possible pour un programmeur et un ingénieur en
électronique ;
\item Avoir une syntaxe robuste et et une forte sémantique.
\end{itemize}

Le code C généré doit être robuste puisque le système d'exploitation dépend de lui.

\section{Le compilateur}

Avec le multi-plateforme, la capacité de réutiliser le code devient un enjeu.
Pour chaque plateforme, le générateur de pilote doit :
\begin{itemize}
\item Être installable sur chaque système supporté ;
\item Être capable d'être lancé en ligne de commande ;
\item Vérifier la syntaxe du DSL ;
\item Vérifier la sémantique de ses entrées ;
\item Utiliser le CodeWorker\cite{CodeWorker} ;
\item Utiliser les outils natifs \`a chaque plateforme.
\end{itemize}

\section{La \BL}

La \BL\ doit :
\begin{itemize}
\item Contenir le code des patrons utilisés par le compilateur ;
\item Être amélioré avec la gestion des bus ;
\item Mise \`a jour avec la nouvelle version du compilateur.
\end{itemize}

\section{Support de l'asynchrone}

Actuellement, \rtx\ supporte seulement la communication synchrone avec les
périphériques. Cependant, la plupart des périphériques communiquent de manière
asynchrone (comme l'USB) et sa gestion dans le DSL est incontournable.

\section{Nouveaux pilotes}

A la fin du projet, les pilotes suivants devront être disponibles :
\begin{itemize}
\item souris USB ;
\item clef de stockage USB ;
\item carte Ethernet ;
\item sondes a travers l'i2c.
\end{itemize}

\section{Licence}

En tant que projet scientifique \rtx\ est distribué selon deux licences open
source :
\begin{description}
\item[GPLv3:] Pour le compilateur, afin de garder une propriété intellectuelle ;
\item[BSD:] Pour la \BL\ afin de rendre possible l'adoption de \rtx\ par des
entreprises.
\end{description}

\chapter{Contrôle du projet}

\section{Statut de l'association}

\rtx\ est une association loi de 1901. Une mise à jour des statuts est
prévue pour fin 2010.

Un compte en banque géré par l'association sera créé afin de subvenir aux
besoins du projet.

Dépenses prévues :
\begin{itemize}
\item Achat de périphériques (souris, carte réseaux, \ldots) ;
\item Voyage (RMLL\footnote{Rencontres Mondiales du Logiciel
Libre(\url{http://rmll.info/}).},
Fosdem\footnote{\url{http://www.fosdem.org/}.}) ;
\item Nom de domaine (rathaxes.org).
\end{itemize}

En tant qu'association, \rtx\ peut recevoir des donations.

\section{Environnement de travail}

\begin{itemize}
\item Le LSE est capable de subvenir à la plupart des besoins en matériel ;
\item Trois systèmes d'exploitation seront utilisés pour les recherches et les
tests, trois machines x86 seront donc nécessaires. Cependant elles pourront
être virtualisées afin d'unifier l'environnement de test ;
\item Les périphériques de test seront obtenus à partir du LSE ou achetés si
besoin ;
\item Le code est hébergé sur GoogleCode à l'adresse :
\url{http://code.google.com/p/rathaxes/} ;
\item Le wiki du google code propose un point de départ pour installer et
tester \rtx\ ;
\item Le code est versionné avec
Mercurial\footnote{\url{http://mercurial.selenic.com/}.} ;
\item Le gestionnaire de bogues (tickets) est public et utilisé pour développer
le projet ;
\item Le code, les tickets ainsi que la documentation doivent être écrits en
anglais ;
\item En plus des courriels Epitech, chaque membre du projet doit être présent
sur le canal IRC suivant: \url{irc://rathaxes@irc.freenode.org}.
\end{itemize}

\section{Planification}

\rtx\ 2012 est la somme de plusieurs étapes :

\begin{enumerate}
\item Étudier des pilotes existant sur chaque plateforme ;
\item Écrire des pilotes sur chaque plateforme ;
\item Définir les concepts à implémenter dans le DSL ;
\item Commencer \`a travailler avec le code de \rtx\ ;
\item Mettre \`a jour le DSL ;
\item Écrire des pilotes avec \rtx\ ;
\item Participer aux RMLL et Fosdem.
\end{enumerate}

%\begin{center}
%\includegraphics[angle=90,scale=0.35]{../images/gantt}
%\end{center}

\rtxbibliography

\end{document}
