\documentclass[francais]{rtxarticle}

\usepackage[utf8]{inputenc}

\title{Documentation utilisateur}
\author{Louis Opter}

\rtxdoctype{UD2}
\rtxdocstatus{Finale}
\rtxdocversion{0.1}

\rtxdochistory
{
0.1 & 05/05/2011 & Louis Opter & Traduction \\
}

\begin{document}

\maketitle

\begin{abstract}
\rtx\ est un Langage de programmation Spécifique à un Domaine (DSL) qui permet
de décrire des pilotes de périphériques. \rtx\ compile vers des modules noyaux
écrits en C pour Linux, Windows et OpenBSD.

La documentation utilisateur est découpée en trois parties, ce document contient
un lien vers chaque document.
\end{abstract}

\section*{Guide d'installation}

Ce document explique comment installer \rtx\ sous Windows ou n'importe quel
Unix. Il explique aussi comment générer votre premier pilote avec \rtx\ et
quelles sont les différentes parties du langage et du projet~:
\begin{itemize}
\item {\small\url{http://rathaxes.googlecode.com/files/firststeps_fr.pdf}}.
\end{itemize}

\section*{Langage -- front-end}

Ce document décrit la partie « description de pilotes » du langage \rtx\
(fichiers \texttt{.rtx})~:
\begin{itemize}
\item {\small\url{http://rathaxes.googlecode.com/files/dsl_frontend_fr.pdf}}.
\end{itemize}

\section*{Langage -- back-end}

Ce document décrit la partie « patron de codes » du langage \rtx\ (fichiers
\texttt{.blt})~:
\begin{itemize}
\item {\small\url{http://rathaxes.googlecode.com/files/dsl_backend_fr.pdf}}.
\end{itemize}

\rtxmaketitleblock

\end{document}
